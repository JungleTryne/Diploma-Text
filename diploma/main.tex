% !TEX root = ../diploma.tex

\section{Введение}

\subsection{Актуальность}
    Идентификация авторов рукописных текстов является актуальной задачей в области компьютерного зрения. Среди сфер использования данной технологии можно выделить анализ исторических документов, обработку рукописных текстов в судебной практике, кластеризацию огромного количества рукописных текстов в образовательной сфере, а также разметку датасета по писателям в автоматическом режиме, что может помочь улучшить качество работы генеративных нейронных сетей, обученных на рукописных текстах.

\subsection{Постановка задачи}    
    Работы по данной теме выделяют онлайн и оффлайн методы рапознавания авторов. Онлайн метод подразумевает обработку рукописного текста, который представлен в виде временных фрагментов штрихов, из которых извлекается уникальная информация о писателе. В свою очередь, оффлайн метод проводит анализ изображения уже написанного рукописного текста, из которого извлекаются признаки, по которым выявляется автор.
    
    Задачу идентификации авторов можно решать в постановке как задачи классификации, так и задачи кластеризации. В случае задачи классификации каждый автор представляется из себя отдельных класс, который модель предсказывает, имея на вход рукописный текст. В случае задачи кластеризации, не зная заранее множество авторов и их количество, рукописные фрагменты разбиваются на кластера, каждый из которых предположительно написал один человек. Стоит отметить, что если задача решена в постановке кластеризации, то она решена в постановке классификации, так как в случае успешной кластеризации, можно сопоставить полученные кластера уже известным классам.

    Данная дипломная работа будет решать задачу идентификации авторов в формулировке оффлайн кластеризации. Имея на входе документы с рукописным текстом, нужно определить количество писателей и кластеризовать тексты по авторам. Документы могут из себя представлять как полноценные тексты на бумаге, так и отдельно написанные от руки слова или предложения. Обученной модели на стадии inference могут подаваться тексты писателей, которых она не видела во время обучения.

\section{Обзор существующих методов}
    Исследования в области идентификации авторов рукописных текстов проводились в течении многих лет, и улучшали постепенно результаты, предлагая различные методы и идеи извлечения и обработки признаков рукописного текста. 
    
    Для выявления признаков из полученного изображения современные научные работы в основном делают выбор на сверточных нейронных сетях. Используются различные архитектуры, включая ResNet-18 (источник), ResNet-50 (источник), VGG (источник). Данные модели показали хорошие результаты в классификационной постановке задачи, где их применяли в качестве энкодеров.

    Научные работы предлагают различные варианты обучения энкодера. Например, некоторые исследования обучают энкодер в паре с полносвязной нейронной сетью, используя функцию потерь CrossEntropy (источник). Также есть исследования в области применения сиамской архитектуры обучения энкодера на данной задаче (источник).  

    Существует также несколько способов извлечения фрагментов из рукописного текста для дальнейшего извлечения признаков. Один из самых простых способов заключается в нарезания рукописного текста на слова или просто на фрагменты определенной ширины (источник). В некоторых работах из рукописного текста извлекаются самые информативные элементы почерка, которые обнаруживаются различными алгоритмами обнаружения углов (corner-detectors), например, HARRIS и FAST. После прохождения через сверточную нейронную сеть, полученные эмбеддинги потом агрегируются различными способами. Например находится среднее арифметическое векторов или используется алгоритм агрегации VLAD. 

    В целях значительного увеличения датасета и, в последствии, улучшения качества обучения, существует идея синтетической генерации датасета рукописных текстов, используя шрифты, похожие на рукописный текст, и применяя аугментацию (источник). 

    Также, в исследованиях распознования лиц применяется техника обучения Metric Learning, которая помогает в той области получить более репрезентативные эмбеддинги. Так, используя функцию потерь ArcFace удалось достичь значительного улучшения результата в задачи классификации фотографий лиц людей (источник). Не исключено, что данный метод хорошо себя может показать и на рукописных текстах.
    
\section{Разработанные решения}
    Исходя из вышеописанных работ, можно составить общую архитектуру решения поставленной задачи. Рукописные тексты сначала проходят через стадию предобработки, во время которой улучшается качество самого рукописного текста, а также происходит его разбивка на фрагменты, либо путем нарезания на слова/части одинаковой ширины, либо путем применения алгоритма нахождения углов для получения максимально репрезентативных элементов почерка.
    Далее, эти фрагменты поступают в энкодер, который представляет из себя сверточную нейронную сеть, в результате чего получаются эмбеддинги. Далее, эти эмбеддинги при необходимости агрегируются в глобальный эмбеддинг фрагмента текста, если ранее был применен corner-detector. Наконец, применяется алгоритм уменьшения размерности эмбеддингов для улучшения качества кластеризации и применяется сам алгоритм кластеризации.

\subsection{Препроцессинг и дальнейшее агрегирование}

На вход энкодеру не подается целое изображение документа рукописного текста, так как в нем может содеражться лишняя информация, и энкодеру может быть сложно извлечь репрезентативные признаки из него. Вместо этого рукописный текст поддаётся предобработке. 

\subsubsection{Нарезание текста на равные фрагменты}



\subsubsection{Corner-detectors}

\subsubsection{Агрегирование средним и VLAD}

\subsection{Обучение энкодера}

Вышеописанная архитектура предполагает, что энкодер уже был обучен выдавать репрезентативные эмбеддинги. Этого можно добиться несколькими способами, которые будут описаны далее в данной главе.

\subsubsection{Auto-encoder}

\subsubsection{Сиамская нейронная сеть}

\subsubsection{Обучение на задаче классификации}

\subsection{Кластеризация}

Существует множество алгоритмов кластеризации, некоторые из которых принимают на вход уже известное количество кластеров, которое в нашем случае является неизвестным.

\subsubsection{Уменьшение размерности}

\subsubsection{Определение количество кластеров}

\subsubsection{Алгоритм кластеризации}

\section{Результаты проведения экспериментов}

\section{Заключение}
