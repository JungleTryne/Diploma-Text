\usepackage{luatex85}
\pdfminorversion=5

\usepackage{fontspec}
\usepackage{polyglossia}
\usepackage[a4paper, 
lmargin=30mm, rmargin=15mm, tmargin=20mm, bmargin=20mm]{geometry}
\usepackage{multirow}

\usepackage{pgfplots}
\pgfplotsset{compat=1.18}

\setdefaultlanguage{russian}
\setotherlanguage{english}

% Согласно требованиям к ВКР
\defaultfontfeatures{Ligatures=TeX}

\setmainfont{Times New Roman}
\setmonofont{Courier New}
\setsansfont{Arial}

\newfontfamily\cyrillicfont{Times New Roman}
\newfontfamily\cyrillicfontsf{Arial}
\newfontfamily\cyrillicfonttt{Courier New}

\newfontfamily\englishfont{Times New Roman}
\newfontfamily\englishfontsf{Arial}
\newfontfamily\englishfonttt{Courier New}

\linespread{1.5}

\usepackage{titlesec}

\titleformat{\section}{\normalfont\fontsize{14}{14}\bfseries}{\thesection}{1em}{}
\titleformat{\subsection}{\normalfont\fontsize{14}{14}\bfseries}{\thesubsection}{1em}{}
\titleformat{\subsubsection}{\normalfont\fontsize{14}{14}\bfseries}{\thesubsubsection}{1em}{}

\setlength{\parindent}{1.25cm}

\renewcommand\thesection{\arabic{section}}

\usepackage[a-1b,mathxmp]{pdfx}
\hypersetup{hidelinks,colorlinks=true,citecolor=BurntOrange,urlcolor=Blue}

\usepackage[backend=biber,
bibencoding=utf8,
sorting=none,
style=gost-numeric,
language=autobib,
autolang=other,
clearlang=true,
defernumbers=true,
sortcites=true,
doi=true,
isbn=true,
]{biblatex}

\renewcommand{\UrlFont}{\small\rmfamily\tt}
\appto{\bibsetup}{\raggedright}

\addbibresource{bibliography/betti.bib}
\bibliography{bibliography/}


\usepackage{amsfonts}

\usepackage{tikz}
\usepackage{pgfplots}
\usepackage{wrapfig}

\usepackage{caption}
\usepackage{subcaption}

\usepackage{amsfonts, amssymb, amsmath, amsthm}
\usepackage{mathtools}
\usepackage{enumerate}
\usepackage{verbatim}

\usepackage{mathrsfs,amsfonts,mathtools}
\usepackage{amsmath}          
\usepackage{amssymb}

\usepackage{amsthm} 
\usepackage{thmtools}

\usepackage{graphicx} 
\graphicspath{{images/}}

\usepackage{multirow}
\usepackage{booktabs}
\usepackage{adjustbox}
\usepackage{tabularx}

\renewcommand{\emptyset}{\varnothing}
\renewcommand{\epsilon}{\varepsilon}
\renewcommand{\phi}{\varphi}
\renewcommand{\kappa}{\varkappa}

\renewcommand*{\maketitle}{
\begin{titlepage}
  \begin{center}
    \linespread{1}
    \small
    Федеральное государственное автономное образовательное учреждение\break высшего образования\par
    <<Московский физико-технический институт \break (национальный исследовательский университет)>>\par
    Физтех-школа Прикладной Математики и Информатики\par
    Кафедра корпоративных информационных систем\par
  \end{center}
%
%  {
%    \small
%    {\bf Направление подготовки / специальность}: 01.03.02 Прикладная математика и информатика\newline
%    {\bf Направленность (профиль) подготовки}: Прикладная математика и компьютерные науки
%  }
%
  {
    \topskip0pt
    \vspace*{\fill}
    \begin{center}
      {\bf\Large ИДЕНТИФИКАЦИЯ АВТОРОВ \\ ДЛЯ РУКОПИСНЫХ ТЕКСТОВ}\par
    \end{center}
    \vspace*{\fill}
  }
%
  \hfill
  \begin{minipage}[t]{8cm}
    {\bf Студент:\newline}
    Мишин Данила Константинович\newline
    \vspace{-3mm}
    %\rule{8cm}{0.15mm}
    %\centerline{\scriptsize\it (подпись студента)}\newline
    \newline
%
    {\bf Научный руководитель:\newline}
    Бегаев Артур Андреевич\newline
    \vspace{-3mm}
    %\rule{8cm}{0.15mm}
    %\centerline{\scriptsize\it (подпись научного руководителя)}
  \end{minipage}

    \vspace*{\fill}
    \begin{center}
      Москва 2023
    \end{center}
\end{titlepage}
}

\DeclareCaptionFormat{myformat}{\itshape#1#2#3}
\captionsetup{format=myformat}